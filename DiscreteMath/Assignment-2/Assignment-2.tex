\documentclass[12pt]{article}
\setlength{\oddsidemargin}{0in}
\setlength{\evensidemargin}{0in}
\setlength{\textwidth}{6.5in}
\setlength{\parindent}{0in}
\setlength{\parskip}{\baselineskip}

\usepackage{amsmath,amsfonts,amssymb,bm}
\usepackage[scale=0.75,top=1cm,bottom=3cm]{geometry}
\begin{document}

\textbf{{\Large Minh Anh Nguyen }}\\~\\
\textbf{{\Large Discrete Math Homework 2}}

\hrulefill

\begin{enumerate}
%--%

\item %Question 1 
  Use truth table to establish the \textit{modus tollens} tautology:
    \[ \left. \begin{array}{r}
      \neg q \\
    p \rightarrow q \end{array} \right\} \Longrightarrow \neg p\] \\
    \textbf{Answer: }
    \begin{displaymath}
      \begin{array}{|c c|c|c|c|c|c|}
      % |c c|c| means that there are three columns in the table and
      % a vertical bar ’|’ will be printed on the left and right borders,
      % and between the second and the third columns.
      % The letter ’c’ means the value will be centered within the column,
      % letter ’l’, left-aligned, and ’r’, right-aligned.
      p & q & \neg q & p \rightarrow q & (\neg q) \wedge (p \rightarrow q) & \neg p & [(\neg q) \wedge (p \rightarrow q)] \rightarrow \neg p\\% Use & to separate the columns
      \hline % Put a horizontal line between the table header and the rest.
      T & T & F & T & F & F & T \\
      T & F & T & F & F & F & T \\
      F & T & F & T & F & T & T \\
      F & F & T & T & T & T & T \\
      \end{array}
      \end{displaymath}
    Since all the values in the last row are True, the argument is a tautology.

\item 
Fill in the reasons in the following proof sequence. Make sure you indicate \\ which step(s) each derivation rule refers to.
\begin{displaymath}
  \begin{array}{l|l c|c|c|c|c|}
  Statements & Reasons \\% Use & to separate the columns
  \hline % Put a horizontal line between the table header and the rest.
  1. q \wedge r & Given \\
  2. \neg(\neg p \wedge q) & Given \\
  3. \neg \neg p \vee \neg q & \textbf{De Morgan's Laws, 2} \\
  4. p \vee \neg q & \textbf{Double Negation, 3} \\
  5. \neg q \vee p & \textbf{Commutative Laws, 4}\\
  6. q \rightarrow p & \textbf{Implication, 5}\\
  7. q & \textbf{Simplification, 1}\\
  8. p & \textbf{Modus Ponens, 6, 7}\\
  \end{array}
\end{displaymath}

\item 
Fill in the reasons in the following proof sequence. Make sure you indicate which step(s) each derivation rule refers to.
\begin{displaymath}
  \begin{array}{l|l c|c|c|c|c|}
  Statements & Reasons \\% Use & to separate the columns
  \hline % Put a horizontal line between the table header and the rest.
  1. (p \wedge q) \rightarrow r & Given \\
  2. \neg (p \wedge q) \vee r & \textbf{Conjunction, 1}  \\
  3. (\neg p \vee \neg q) \vee r& \textbf{De Morgan's Laws, 2} \\
  4. \neg p \vee (\neg q \vee r) & \textbf{Associative Laws, 3} \\
  5. p \rightarrow (\neg q \vee r) & \textbf{Implication, 4}\\
  \end{array}
\end{displaymath}

\newpage
\item 
Is the proof in Exercise 2 reversible? Why or why not?\\
\textbf{Answer: No, the proof is not reversible because it used Modus Ponens and Simplification, which are one-way operations.}

\item
Is the proof in Exercise 3 reversible? Why or why not?\\
\textbf{Answer: Yes, the proof is reversible because it only used Conjunction, De Morgan's Laws, Associative, and Implication, which are reversible operations.}

\item
Fill in the reasons in the following proof sequence. Make sure you indicate which step(s) each derivation rule refers to.
\begin{displaymath}
    \begin{array}{l|l c|c|c|c|c|}
    Statements & Reasons \\% Use & to separate the columns
    \hline % Put a horizontal line between the table header and the rest.
    1. p \wedge (q \vee r) & Given \\
    2. \neg (p \wedge q) & Given \\
    3. \neg p \vee \neg q & \textbf{De Morgan's Laws, 2} \\
    4. \neg q \vee \neg p & \textbf{Commutative Laws, 3} \\
    5. q \rightarrow \neg p & \textbf{Implication, 4}\\
    6. p & \textbf{Simplification, 1}\\ 
    7. \neg (\neg p) & \textbf{Double Negation, 6}\\
    8. \neg q & \textbf{Modus Tollens, 5, 7} \\
    9. (q \vee r) \wedge p & \textbf{Commutative Laws, 1}\\
    10. q \vee r & \textbf{Simplification, 9}\\
    11. r \vee q & \textbf{Commutative Laws, 10}\\
    12. \neg (\neg r) \vee q & \textbf{Double Negation, 11}\\
    13. \neg r \rightarrow q & \textbf{Implication, 12}\\
    14. \neg(\neg r) & \textbf{Modus Tollens, 8, 13}\\
    15. r & \textbf{Double Negation, 14}\\
    16. p \wedge r& \textbf{Conjunction, 6, 15}\\
    \end{array}
  \end{displaymath}

\item 
Justify each conclusion with a derivation rule.
  \begin{enumerate}
    \item If Joe is artistic, he must also be creative. Joe is not creative. Therefore, Joe is not artistic.\\
    \textbf{Answer: Modus Tollens}
    \item Lingli is both athletic and intelligent. Therefore, Lingli is athletic.\\
    \textbf{Answer: Simplification}
    \item If Monique is 18 years old, then she may vote. Monique is 18 years old. Therefore, Monique may vote.\\
    \textbf{Answer: Modus Ponens}
    \item Marianne has never been north of Saskatoon or south of Santo Domingo. In other words, she has never been north of Saskatoon and she has never been south of Santo Domingo.\\
    \textbf{Answer: De Morgan's Laws}
  \end{enumerate}

\newpage 
\item Which derivation rule justifies the following argument?
\begin{center}  
  If n is a multiple of 4, then n is even. \\ However, n is not even. Therefore, n is not a multiple of 4.
\end{center} 
\textbf{Answer: Modus Tollens}

\item Let x and y be integers. Given the statement. What statement follows by the Implication rule?
\begin{center}
  x $>$ y or x is odd.
\end{center}
\textbf{Answer: The statement follows by Implication Law is:\\ \textquotedblleft If x is not larger than y, then x is odd.\textquotedblright}

\item Let Q be a quadrilateral. Given the statements. What statement follows by modus tollens?
\begin{center}
  If Q is a rhombus, then Q is a parallelogram.\\
  Q is not a parallelogram.
\end{center}
\textbf{Answer: The statement follows by Modus Tollens Law is: \\ \textquotedblleft Q is not a rhombus.\textquotedblright}

\item Let x and y be numbers. Simplify the following statement using De Morgan’s laws and double negation.
\begin{center}
  It is not the case that x is not greater than 3 and y is not found.
\end{center}
\textbf{Answer:} Assign the following statement:
\begin{center}
  \textit{p = \textquotedblleft x is greater than 3.\textquotedblright}\\
  \textit{q = \textquotedblleft y is found.\textquotedblright}
\end{center}
\begin{displaymath}
  \begin{array}{l|l c|c|c|c|c|}
  Statements & Reasons \\% Use & to separate the columns
  \hline % Put a horizontal line between the table header and the rest.
  1. \neg (\neg p \wedge \neg q) & Given \\
  \bm{2. \neg (\neg p) \vee \neg (\neg q)} & \textbf{De Morgan's Laws, 1} \\
  \bm{3. p \vee q} & \textbf{Double Negation, 2} \\
  \end{array}
\end{displaymath}

%--%
\end{enumerate}

\newpage
\begin{enumerate}
\setcounter{enumi}{16}
\item Write a proof sequence for the following assertion. Justify each step.
\[ \left. \begin{array}{r}
      p \rightarrow q \\
    p \wedge r \end{array} \right\} \Longrightarrow q \wedge r \] \\ 
\textbf{Answer: } \\ Proof: 
\begin{displaymath}
  \begin{array}{l|l c|c|c|c|c|}
  Statements & Reasons \\% Use & to separate the columns
  \hline % Put a horizontal line between the table header and the rest.
  1. p \rightarrow q & Given \\
  2. p \wedge r & Given \\
  3. p & \text{Simplification, 2} \\
  4. q & \text{Modus Ponens, 1, 3} \\
  5. r & \text{Simplification, 2} \\
  6. q \wedge r & \text{Conjunction, 4, 5} \\
  \end{array}
\end{displaymath}

~\\

\item 
Write a proof sequence for the following assertion. Justify one of the steps in your proof using the result of Example 1.8.
\[ \left. \begin{array}{r}
      \neg (a \wedge \neg b) \\
      \neg b \end{array} \right\} \Longrightarrow \neg a \] \\ 
\textbf{Answer: } \\ Proof: 
      \begin{displaymath}
        \begin{array}{l|l c|c|c|c|c|}
        Statements & Reasons \\% Use & to separate the columns
        \hline % Put a horizontal line between the table header and the rest.
        1. \neg(a \wedge \neg b) & Given \\
        2. \neg b & Given \\
        3. \neg a \vee \neg (\neg b) & \text{De Morgan's Laws, 1} \\
        4. \neg a \vee b & \text{Double Negation, 3} \\
        5. a \rightarrow b & \text{Implication, 4} \\
        6. \neg a & \text{Modus Tollens, 2, 5} \\
        \end{array}
      \end{displaymath}
\end{enumerate}
\end{document}