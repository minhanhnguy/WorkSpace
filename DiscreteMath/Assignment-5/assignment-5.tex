\documentclass[12pt]{article}
\setlength{\oddsidemargin}{0in}
\setlength{\evensidemargin}{0in}
\setlength{\textwidth}{6.5in}
\setlength{\parindent}{0in}
\setlength{\parskip}{\baselineskip}

\usepackage{amsmath,amsfonts,amssymb,bm,graphics,pgfplots,framed,dsfont}
\usepackage[scale=0.75,top=1cm,bottom=3cm]{geometry}

\begin{document}

\textbf{Minh Anh Nguyen }\\
\textbf{Discrete Mathematics\hfill Assignment-5}

\hrulefill

\begin{enumerate}
      \item Let X be a set with four elements. Represent the identity function $1_X$ of Example 2.22 with a directed graph in two different ways:
      \begin{enumerate}
            \item as an f-graph with four vertices\\
            Let:
                  \[X = \{a,b,c,d\}\]
            The identity function:
            \[X \text{~~~~~~} X\]
            \[a \longrightarrow a\]
            \[b \longrightarrow b\]
            \[c \longrightarrow c\]
            \[d \longrightarrow d\]
            \item with eight vertices, four for the domain and four for the codomain
            Let:\\
                  Domain: \[X = \{a,b,c,d\}\]
                  Codomain: \[Y = \{a,b,c,d\}\]
            \[X \text{~~~~~~} Y\]
            \[a \longrightarrow a\]
            \[b \longrightarrow b\]
            \[c \longrightarrow c\]
            \[d \longrightarrow d\]
      \end{enumerate}

      \item See Definitions 2.3 and 2.4. Write the definitions of one-to-one and onto in terms of predicate logic.\\
      One-to-one definitions in predicate logic:
        \[(\forall x,y \in X)[(f(x) = f(y)) \longrightarrow (x = y)]\]
      Onto definitions in predicate logic:
        \[(\forall y \in Y)(\exists x \in X)[f(x) = y]\]
      \newpage
      \item Show that the function of Example 2.20 is not one-to-one.\\
      Let:
        \[S1 = \{-1,0,1\}\]
        \[S2 = \{0\}\]
      The sum of $S1$ is:
        \[-1 + 0 + 1 = 0\]
      The sum of $S2$ is:
        \[0\]
      \[s(S1) = s(S2) \text{ but } S1 \neq S2\]
      Hence, the function is not one-to-one.
      \item Show that the function of Example 2.20 is onto.\\
      To show the function is onto, for every y there must be \{X\} that the sum of it is y.
      \[y \in Z\]
      and because \{X\} is the set of all nonempty finite sets of integers.
      \[{X} \subset Z\] 
      Hence, we can always choose a set with one value y so:
      \[s(\{y\}) = y\]
      Therefore, the function is onto.
      \item Several languages are spoken in India; let L be the set of all such languages, and let U be the set of all residents of India. Explain why the proposed function f: U → L defined by
            $f(u) = \text{ the language that $u$ speaks.}$
      is not well defined.\\
      Because a person can speaks several languages there can be more than one $f(u)$ with one $u$.\\
      Hence, the function is not well defined.
      \item Let P be a set of people, and let Q be a set of occupations. Define a function f: P → Q by setting f(p) equal to p’s occupation. What must be true about the people in P for f to be a well-defined function?\\~\\
      If f is well defined, for each of the people in P, they must have exactly one occupation in P.
      \item Is the function of Example 2.23 onto? Why or why not? Is it one-to-one? Why or why not?\\
      The function of Example 2.23 is not onto because for all people in P, there are males and females that don't want to give birth.\\
      The function of Example 2.23 is also not one-to-one because there are siblings that have the same birth mothers in P.
      \newpage
      \item Consider Example 2.23. Let y be some person. What is the relationship of $(m \circ m) (y)$ to y?
      \[(m \circ m) (y) = m(m(y))\]
      This means a birth mother of the birth mother of y. Which is y's grandma.
      \item Is the function depicted in Figure 2.8 onto? Why or why not?\\
      \begin{figure}[ht]
            \centering
                 \includegraphics[width=1.0\textwidth]{img/Figure2-8.png}
                  Figure 2.8
                  \label{normal_case}
      \end{figure}\\
      The function depicted in Figure 2.8 is not onto because there 2 y elements in Y that doesn't have any x in X that f(x) = y.
      \item Is the function depicted in Figure 2.9 one-to-one? Why or why not?\\
      \begin{figure}[ht]
            \centering
                 \includegraphics[width=1.0\textwidth]{img/Figure2-9.png}
                  Figure 2.9
                  \label{normal_case}
      \end{figure}\\
      The function depicted in Figure 2.9 is not one-to-one because there are 2 values of x in X with the same y in Y that f(x) = y.
      \item Explain why the proof in Example 2.28 could not be used to prove that the function in Example 2.26 is onto.\\
      Because in Example 2.28, $f:\mathds{R} \longrightarrow \mathds{R}$ so that there will always be an $x$ for $y$ in $\mathds{R}$.
      In Example 2.26, $f:\mathds{Z} \longrightarrow \mathds{Z}$. If y = 6, 2x + 1 = 6 doesn't have any solution in $\mathds{Z}$.
      \newpage
      \item Consider the situation of Example 2.30. Describe a different one-to-one correspondence $g: Y \longrightarrow X$. Show that your function is both one-to-one and onto.\\
      Let:
            \[X: \text{ the set of all points of intersection of the lines in the interior of the circle.}\]
            \[Y = \{A,B,C,D\}: \text{ the sets of all sets of the points on the circle.}\]
            \[g: Y \longrightarrow X\]
            \[g(\{A,B,C,D\}) = H \text{ with H is the intersection of the line AB and CD.}\]
      Onto and One-to-one Proof:\\
            Because H is made of the intersection of 2 lines (there are no 3 lines intersection), every H is linked with a different sets of 4 points. Hence, the function g is both one-to-one and onto.
      \item Consider the negation function \( n: \{T, F\} \rightarrow \{T, F\} \) given by \( n(x) = \neg x \). Is \( n \) a one-to-one correspondence? What is \( n^{-1} \)?\\
            The function n(x) is one-to-one correspondence because there are only 2 elements in both the input and output, and:
                  \[n(T) = F\]
                  \[n(F) = T\]
            Therefore, every output elements are linked with a unique input elements. Hence, the function is both onto and one-to-one.\\
            The function \(n^{-1}(x) = \neg x\) which is also \(n(x)\).
      \item Define a function \(f: \mathds{R} \longrightarrow \mathds{R}\) by the formula \(f(x) = 3x - 5\).
      \begin{enumerate}
            \item Prove that f is one-to-one.\\
            Let $a,b \in \mathds{R}$:
                  \[f(a) = f(b)\]
                  \[3a - 5 = 3b - 5\]
                  \[3a = 3b\]
                  \[a = b\]
            Hence, the function is one-to-one.            
            \item Prove that f is onto.\\
            Let $x = \frac{y+5}{3}$ with $x,y \in \mathds{R}$: 
                  \[f(x) = f(\frac{y+5}{3})\]
                  \[f(x) = 3\times \frac{y+5}{3} - 5\]
                  \[f(x) = y + 5 - 5\]
                  \[f(x) = y\]
            Thus $f$ is onto.
            \newpage
      \end{enumerate}
      \item Let Z denote the set of integers, let $Z^*$ denote the set of nonzero integers, and let Q be the set of all rational numbers. Define a function $g : Z \times Z^* \rightarrow Q \text{ by } g(a, b) = a/b$. Explain why g is not one-to-one. Be specific.\\
      For $g(2,1)$:
            \[g(2,1) = \frac{2}{1} = 2\]
      For $g(4,2)$:
            \[g(4,2) = \frac{4}{2} = 2\]
      Therefore, $g(2,1) = g(4,2)$, but $(2,1) \neq (4,2)$.\\ Hence, the function $g$ is not one-to-one.
      \setcounter{enumi}{16}
      \item Define a function $f: \mathds{Z} \rightarrow \mathds{Z} \times \mathds{Z}$ by $f(x) = (2x+3,x-4)$.
      \begin{enumerate}
            \item Is f one-to-one? Prove or disprove.\\
                  Let:
                        \[f(a) = f(b)\]
                        \[(2a+3,a-4) = (2b+3,b-4)\]
                        \[2a+3 = 2b+3 \text{ and } a-4 = b-4\]
                        \[2a = 2b \text{ and } a = b\]
                        \[a = b \text{ and } a = b\]
                        \[a = b\]
                  Therefore, $(f(a) = f(b)) \rightarrow (a = b)$\\
                  Hence, the function f is one-to-one.
            \item Does f map Z onto $Z \times Z$? Prove or disprove.\\
                  Let choose $(3,3) \in \mathds{Z} \times \mathds{Z}$:
                     \[f(x) = (3,3)\]   
                     \[(2x+3, x-4) = (3,3)\]   
                     \[2x+3 = 3 \text{ and } x-4 = 3\]   
                     \[2x=0 \text{ and } x = 7\]   
                     \[x=0 \text{ and } x = 7\]   
                  Therefore, no single x satisfies both equations. There is no such x that maps to (3,3).
                  Hence, the function is not onto.
      \end{enumerate}
      \item Define a map $t: \mathds{R} \times \mathds{R} \rightarrow \mathds{R} \times \mathds{R}$ by $t(a,b) = (a+b,a-b)$. Prove that t is a one-to-one correspondence.\\
      Let $\{a,b,c,d\} \subset \mathds{R}$:
            \[t(a,b) = t(c,d)\]
            \[(a+b,a-b) = (c+d,c-d)\]
            \[a+b = c+d \text{ and } a-b = c-d\]
      Add both function together:
            \[2a = 2c\]
            \[a = c\]
      Since $a=c$:
            \[b=d\]
      Therefore, $[t(a,b) = t(c,d)] \rightarrow (a,b) = (c,d)$. Hence, the function t is one-to-one.
      \item Let $X$ be a set. Define a map $d: X \rightarrow X \times X \text{ by } d(x) = (x, x)$.
            \begin{enumerate}
                  \item Is d one-to-one? Prove or disprove.\\
                  The function d is one-to-one since for every output pair $(x,x)$ with $x \in X$, there is only one input $x \in X$ that maps to it.
                  \item Is d onto? Prove or disprove.\\
                  The function d is not onto since for output pair $(x_1,x_2)$ with $x_1,x_2 \in X$ that $x1 \neq x2$. There are no satisfies x that maps into that pair.

            \end{enumerate}
\end{enumerate}

\end{document}