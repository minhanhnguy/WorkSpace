\documentclass[12pt]{article}
\setlength{\oddsidemargin}{0in}
\setlength{\evensidemargin}{0in}
\setlength{\textwidth}{6.5in}
\setlength{\parindent}{0in}
\setlength{\parskip}{\baselineskip}

\usepackage{amsmath,amsfonts,amssymb,bm,graphics,pgfplots,framed,dsfont}
\usepackage[scale=0.75,top=1cm,bottom=3cm]{geometry}

\begin{document}

\textbf{Minh Anh Nguyen }\\
\textbf{Calculus 1 Assignment-8}\\
\textbf{Section: 04}\\
\textbf{TA's name: Arthur Huey}

\hrulefill

\begin{enumerate}
\setcounter{enumi}{1}
    \item Find two numbers whose difference is 100 and whose product is a minimum.\\
    Let calls that two numbers x and y.
    \[x + y = 100\]
    \[y = 100 - x\]
    Product is a minimum:
    \[P = xy\]
    \[P = x(100-x)\]
    \[P = 100x - x^2\]
    \[P' = 100 - 2x = 0\]
    \[x = 50\]
    Since x = 50:
    \[y = 50\]
    Hence, these two numbers are both 50.
    \setcounter{enumi}{4}
    \item What is the maximum vertical distance between the line \(y=x+2\) and the parabola \(y=x^2\) for $-1 \leq x \leq 2$?\\
    Vertical Distance:
    \[D = |y_{parabola} - y_{line}| = |x^2-x-2|\]
    \[D = |x^2-x-2|\]
    \[D' = |2x-1| = 0\]
    \[x = \frac{1}{2}\]
    Hence, the maximum vertical distance is:
    \[D = |\frac{1}{2}^2 - \frac{1}{2} - 2| =  \frac{9}{4}\]
    \setcounter{enumi}{8}
    \item A model used for the yield $Y$ of an agricultural crop as a function of the nitrogen level $N$ in the soil (measured in appropriate units) is
    \[Y = \frac{kN}{1+N^2}\]
    where $k$ is a positive constant. What nitrogen level gives the best yield?
    \[Y' = \frac{(kN)'(1+N^2) - (1+N^2)'(kN)}{(1+N^2)^2}\]
    \[Y' = \frac{k(1+N^2) - 2N(kN)}{(1+N^2)^2}\]
    \[Y' = \frac{k+kN^2 - 2kN^2}{(1+N^2)^2}\]
    \[Y' = \frac{k -kN^2}{(1+N^2)^2} = 0\]
    \[k-kN^2 = 0\]
    \[kN^2 = k\]
    \[N^2 = 1\]
    \[N = \pm 1\]
    Nitrogen level cannot go below 0.\\
    Hence, $N = 1$ is the best nitrogen level for the best yield.
    \setcounter{enumi}{12}
    \item A farmer wants to fence in an area of 1.5 million square feet in a rectangular field and then divide it in half with a fence parallel to one of the sides of the rectangle. How can he do this so as to minimize the cost of the fence?\\
    The cost of the fence related to the length of the fence in general:\\
    Let x = the length of the two sides of the fence;\\
    Let y = the length of the two sides of the fence and the length of the fence in the middle;\\
    The area of the farm:
    \[A = xy = 1,500,000 (feet^2)\]
    \[y = \frac{1,500,000}{x}\]
    The total length of the fence is:
    \[L = 2x + 3y\]
    \[L = 2x + 3\frac{1,500,000}{x}\]
    \[L = 2x + \frac{4,500,000}{x}\]
    \[L' = 2 - \frac{4,500,000}{x^2} = 0\]
    \[2 - \frac{4,500,000}{x^2} = 0\]
    \[2x^2 - 4,500,000 = 0\]
    \[2x^2 = 4,500,000\]
    \[x^2 = 2,250,000\]
    \[x^2 = 2,250,000\]
    \[x = \pm 1,500\]
    Hence, x = 1,500 since the length cannot be negative.
    \[y = 1000\]
    Hence, the dimensions of the rectangle are 1500ft and 1000ft with the middle is 1000ft.
    \setcounter{enumi}{18}
    \item If $1200cm^2$ of material is available to make a box with a square base and an open top, find the largest possible volume of the box.\\
    The width of the square base is x;\\
    The height of the box is y;\\
    Hence, the area of material available is:
    \[S = x^2 + 4xy = 1200\]
    \[4xy = 1200 - x^2\]
    \[y = \frac{1200 - x^2}{4x}\]
    The volume of the box is:
    \[V = x^2y\]
    \[V = x^2\frac{1200 - x^2}{4x}\] 
    \[V = \frac{1200x - x^3}{4}\] 
    \[V = 300x - \frac{x^3}{4}\] 
    \[V' = 300 - \frac{3x^2}{4} = 0\] 
    \[\frac{3x^2}{4} = 300\]
    \[x^2 = 400\]
    \[x = \pm 20\]
    Hence, x = 20 since the length cannot be negative.
    \[y = 10\]
    Therefore, the largest volume of the box is:
    \[V = x^2y = (20)^2 \times 10 = 4000\]
    \setcounter{enumi}{24}
    \newpage
    \item Find the point on the line y = 2x + 3 that is closet to the origin.\\
    The distance from the origin to the point is:
    \[D = \sqrt{x^2 + y^2}\]
    \[D = \sqrt{x^2 + (2x+3)^2}\]
    \[D = \sqrt{x^2 + 4x^2 + 12x + 9}\]
    \[D = \sqrt{5x^2 + 12x + 9}\]
    \[D' = \frac{10x + 12}{2\sqrt{5x^2 + 12x + 9}} = 0\]
    Since $\sqrt{5x^2 + 12x + 9} \geq 0$ with all x.
    \[10x + 12 = 0\]
    \[x = -\frac{12}{10}\]
    Hence, the closet distance from the origin to the point is:
    \[D = \sqrt{5x^2 + 12x + 9} = \sqrt{5(-\frac{12}{10})^2 + 12(-\frac{12}{10}) + 9} = \frac{3\sqrt{5}}{5}\]

\end{enumerate}

Section 3.8:
\begin{enumerate}
    \setcounter{enumi}{2}
    \item Suppose the tangent line to the curve $y = f(x)$ at the point $(2,5)$ has the equation $y = 9-2x$. If Newton’s method is used to locate a solution of the equation f(x) = 0 and the initial approximation is $x_1 = 2$, find the second approximation $x_2$.
    \[f'(x) = (9-2x)'\]
    \[f'(x) = -2\]
    \[x_2 = x_1 - \frac{f(x_1)}{f'(x_1)} = 2 - \frac{f(2)}{f'(2)} = 2 - \frac{9 - 2\times 2}{-2} = 4.5\]
    \setcounter{enumi}{5}
    \item Use Newton’s method with the specified initial approximation $x_1$ to find $x_3$, the third approximation to the solution of the given equation. (Give your answer to four decimal places.)
    \[2x^3 - 3x^2 + 2 = 0, x_1 = -1\]
    \[f(x) = 2x^3 - 3x^2 + 2 \]
    \[f'(x) = 6x^2 - 6x\]
    Newton's method:
    \[x_2 = x_1 - \frac{f(x_1)}{f'(x_1)}\]
    \[x_2 = -1 - \frac{f(-1)}{f'(-1)} = -1 - \frac{2(-1)^3 - 3(-1)^2 + 2}{6(-1)^2 - 6(-1)} = -\frac{3}{4}\]
    \[x_3 = x_2 - \frac{f(x_2)}{f'(x_2)}\]
    \[x_3 = -3/4 - \frac{f(-3/4)}{f'(-3/4)} = -3/4 - \frac{2(-3/4)^3 - 3(-3/4)^2 + 2}{6(-3/4)^2 - 6(-3/4)} = -\frac{43}{63}\]
    \setcounter{enumi}{10}
    \item Use Newton’s method to approximate the given number correct to eight decimal places.\\
    Let:
    \[\sqrt[4]{75} = x\]
    \[75 = x^4\]
    \[x^4 - 75 = 0\]
    Call this f(x).
    \[f(2) = -59\]
    \[f(3) = 6\]
    Hence, there is a solution in the interval (2,3).
    \[f'(x) = 4x^3\]
    Choose $x_1 = 2.5$.
    \[x_2 = x_1 - \frac{f(x_1)}{f'(x_1)} = 2.5 - \frac{f(2.5)}{f'(2.5)} = 2.5 - \frac{-35.9375}{62.5} = 3.075\]
    \[x_3 = x_2 - \frac{f(x_2)}{f'(x_2)} = 3.075 - \frac{f(3.075)}{f'(3.075)} = 3.075 - \frac{14.40884414}{116.3041875} = 2.951110702\]
    \[x_4 = x_3 - \frac{f(x_3)}{f'(x_3)} = 2.951110702 - \frac{f(2.951110702)}{f'(2.951110702)} = 2.951110702 - \frac{0.8476281147}{102.8055343} = 2.942865736\] 
    \[x_5 = = x_4 - \frac{f(x_4)}{f'(x_4)} = 2.942865736 - \frac{f(2.942865736)}{f'(2.942865736)}\] 
    \[ = 2.942865736 - \frac{0.003545609415}{101.9462692} \approx 2.94283096\]    
    \setcounter{enumi}{12}
    \item
    \[3x^4 - 8x^3 + 2 = 0, [2,3]\]
    \begin{enumerate}
        \item Explain how we know that the given equation must have a solution in the given interval.
        \[f(2) = -14\]
        \[f(3) = 29\]
        Therefore, $f(2) < 0$ and $f(3) > 0$.\\
        Hence, the function has at least one solution in the interval (2,3).
        \item Use Newton’s method to approximate the solution correct to six decimal places.\\
        Choose $x_1 = 2.5$.
        \[f'(x) = 12x^3 - 24x^2\]
        \[x_2 = x_1 - \frac{f(x_1)}{f'(x_1)} = 2.5 - \frac{f(2.5)}{f'(2.5)} = 2.5 - \frac{-5.8125}{37.5} = 2.655\]
        \[x_3 = x_2 - \frac{f(x_2)}{f'(x_2)} = 2.655 - \frac{f(2.655)}{f'(2.655)} = 2.655 - \frac{1.344969352}{55.4053365} = 2.630724913\]
        \[x_4 = x_3 - \frac{f(x_3)}{f'(x_3)} = 2.630724913 - \frac{f(2.630724913)}{f'(2.630724913)}\]
        \[ 2.630724913 - \frac{0.03688007582}{52.38079756} \approx 2.630021\]
        
    \end{enumerate}

    

\end{enumerate}

\end{document}