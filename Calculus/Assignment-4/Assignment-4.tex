\documentclass[12pt]{article}
\setlength{\oddsidemargin}{0in}
\setlength{\evensidemargin}{0in}
\setlength{\textwidth}{6.5in}
\setlength{\parindent}{0in}
\setlength{\parskip}{\baselineskip}

\usepackage{amsmath,amsfonts,amssymb,bm,graphics,pgfplots,framed,dsfont}
\usepackage[scale=0.75,top=1cm,bottom=3cm]{geometry}

\begin{document}

\textbf{Minh Anh Nguyen }\\
\textbf{Assignment 4}\\
\textbf{Section: 04}\\
\textbf{TA's name: Arthur Huey}

\hrulefill

Section 2.3\\~\\

\begin{enumerate}
      \item Differentiate the function.
            \[g(x) = 4x + 7\]
            \[{\displaystyle \frac{d}{dx} g(x) = \frac{d}{dx}(4x+7)}\]
            \[\boxed{{\displaystyle \frac{d}{dx} g(x) = 4}}\]
            \setcounter{enumi}{6}
      \item Differentiate the function.
            \[{\displaystyle f(x) = x^{3/2} + x^{-3}}\]
            \[{\displaystyle \frac{d}{dx}f(x) = \frac{d}{dx}(x^{3/2} + x^{-3})}\]
            \[\boxed{{\displaystyle \frac{d}{dx}f(x) = \frac{3}{2}x^{1/2} -3 x^{-4}}}\]
            \setcounter{enumi}{10}
      \item Differentiate the function.
            \[{\displaystyle y = 2x + \sqrt{x}}\]
            \[{\displaystyle \frac{dy}{dx} = \frac{d}{dx}(2x + \sqrt{x})}\]
            \[\boxed{\displaystyle \frac{dy}{dx} = 2 + \frac{1}{2\sqrt{x}}}\]
            \setcounter{enumi}{14}
      \item Differentiate the function after first rewriting the function in a different form. (Do not use the Product or Quotient Rules.)
            \[f(x) = x^3(x+3)\]
            \[f(x) = x^4+3x^3\]
            \[{\displaystyle \frac{d}{dx}f(x) = \frac{d}{dx}(x^4+3x^3)}\]
            \[\boxed{\displaystyle \frac{d}{dx}f(x) = 4x^3+9x^2}\]
            \setcounter{enumi}{17}
      \item Differentiate the function after first rewriting the function in a different form. (Do not use the Product or Quotient Rules.)
            \[{\displaystyle y = \frac{\sqrt{x} + x}{x^2}}\]
            \[{\displaystyle y = \frac{\sqrt{x}}{x^2} + \frac{x}{x^2}}\]
            \[{\displaystyle y = x^{-\frac{3}{4}} + x^{-1}}\]
            \[{\displaystyle \frac{dy}{dx} = \frac{d}{dx}(x^{-\frac{3}{4}} + x^{-1})}\]
            \[\boxed{\displaystyle \frac{dy}{dx} = -\frac{3}{4}x^{-\frac{7}{4}} - x^{-2}}\]
            \setcounter{enumi}{29}
      \item Use the Product Rule to find the derivative of the function.
            \[y = (10x^2 + 7x -2)(2-x^2)\]
            \[{\displaystyle \frac{d}{dx} y = \frac{d}{dx}[(10x^2 + 7x -2)(2-x^2)]}\]
            \[{\displaystyle \frac{d}{dx} y = \frac{d}{dx}(10x^2 + 7x -2)(2-x^2) + (10x^2 + 7x - 2)\frac{d}{dx}(2-x^2)}\]
            \[{\displaystyle \frac{d}{dx} y = (20x + 7)(2-x^2) + (10x^2 + 7x - 2)(-2x)}\]
            \[{\displaystyle \frac{d}{dx} y = 40x - 20x^3 + 14 - 7x^2 -20x^3 - 14x^2 + 4x}\]
            \[\boxed{\displaystyle \frac{d}{dx} y = -40x^3 -21x^2 + 44x + 14}\]
            \setcounter{enumi}{34}
      \item Use the Quotient Rule to find the derivative of the function.
            \[{\displaystyle g(t) = \frac{3-2t}{5t+1}}\]
            \[{\displaystyle \frac{d}{dx} g(t) = \frac{d}{dx}(\frac{3-2t}{5t+1})}\]
            \[{\displaystyle \frac{d}{dx} g(t) = \frac{\frac{d}{dx}(3-2t)(5t+1) - (3-2t)\frac{d}{dx}(5t+1)}{(5t+1)^2}}\]
            \[{\displaystyle \frac{d}{dx} g(t) = \frac{-2(5t+1) - 5(3-2t)}{(5t+1)^2}}\]
            \[{\displaystyle \frac{d}{dx} g(t) = \frac{-10t - 2 - 15 + 10t}{(5t+1)^2}}\]
            \[\boxed{\displaystyle \frac{d}{dx} g(t) = \frac{- 17}{(5t+1)^2}}\]
            \setcounter{enumi}{41}
            \newpage
      \item Differentiate.
            \[{\displaystyle y = \frac{(u+2)^2}{1-u}}\]
            \[{\displaystyle \frac{d}{du} y = \frac{d}{du}(\frac{(u+2)^2}{1-u}})\]
            \[{\displaystyle \frac{d}{du} y = \frac{\frac{d}{du}(u+2)^2(1-u) - (u+2)^2\frac{d}{du}(1-u)}{(1-u)^2}}\]
            \[{\displaystyle \frac{d}{du} y = \frac{2(u+2)(1-u) + (u+2)^2}{(1-u)^2}}\]
            \[{\displaystyle \frac{d}{du} y = \frac{(2u+4)(1-u) + u^2 + 4u + 4}{(1-u)^2}}\]
            \[{\displaystyle \frac{d}{du} y = \frac{2u-2u^2+4-4u + u^2 + 4u + 4}{1-2u+u^2}}\]
            \[\boxed{\displaystyle \frac{d}{du} y = \frac{-u^2 + 2u + 8}{u^2 - 2u + 1}}\]
            \setcounter{enumi}{58}
      \item Find an equation of the tangent line to the curve at the given point.
            \[{\displaystyle y = \frac{2x}{x + 1}}, (1,1)\]
            \[{\displaystyle \frac{d}{dx}y = \frac{d}{dx} (\frac{2x}{x + 1})}\]
            \[{\displaystyle \frac{d}{dx}y = \frac{\frac{d}{dx}(2x)(x+1) - (2x)\frac{d}{dx}(x+1)}{(x + 1)^2}}\]
            \[{\displaystyle \frac{d}{dx}y = \frac{2(x+1) - (2x)}{(x + 1)^2}}\]
            \[{\displaystyle \frac{d}{dx}y = \frac{2x+ 2 - 2x}{x^2 + 2x + 1}}\]
            \[\boxed{\displaystyle \frac{d}{dx}y = \frac{2}{x^2 + 2x + 1}}\]
            \setcounter{enumi}{74}
      \item Biologists have proposed a cubic polynomial to model the length $L$ of Alaskan rockfish at age $A$:
            \[L = 0.0155A^3 - 0.372A^2 + 3.95A + 1.21\]
            where $L$ is measured in inches and $A$ in years. Calculate ${\displaystyle \frac{dL}{dA}|_{A = 12}}$ and interpret your answer.
            \[{\displaystyle \frac{dL}{dA} = \frac{dL}{dA}(0.0155A^3 - 0.372A^2 + 3.95A + 1.21)}\]
            \[{\displaystyle \frac{dL}{dA} = 0.0465A^2 - 0.744A + 3.95}\]
            \[\boxed{\displaystyle \frac{dL}{dA}|_{A = 12} = 0.0465 \times 12^2 - 0.744\times12 + 3.95 = 1.718}\]
            \noindent\fbox{
                  \parbox{390px}{
                        The length of Alaskan rockfish at age 12 is increasing by 1.718 inches per year.
                  }
            }
            \setcounter{enumi}{86}
      \item Find the points on the curve $y = x^3 + 3x^2 - 9x + 10$ where the tangent is horizontal.
            \[{\displaystyle \frac{dy}{dx} = \frac{d}{dx}(x^3 + 3x^2 - 9x + 10)}\]
            \[{\displaystyle \frac{dy}{dx} = 3x^2 + 6x - 9}\]
            The tangent line is horizontal when the slope is zero.
            \[{\displaystyle 3x^2 + 6x - 9 = 0}\]
            \[x = 1 \text{~} \vee \text{~} x = -3\]
            For x = 1, then:
            \[y = 1^3 + 3 \times 1^2 - 9 \times 1 + 10 = 5\]
            Therefore, the first point is (1, 5)\\~\\
            For x = -3, then:
            \[y = (-3)^3 + 3 \times (-3)^2 - 9 \times (-3) + 10 = 37\]
            Therefore, the second point is (-3, 37).\\
            Hence, there are two points in which the tangent line is horizontal: 
            \[\boxed{(1,5),(-3,37)}\]
\end{enumerate}

Section 2.4:
\begin{enumerate}
      \item Differentiate.
            \[f(x) = 3\sin(x) - 2\cos(x)\]
            \[{\displaystyle \frac{d}{dx}f(x) = \frac{d}{dx}(3\sin(x) - 2\cos(x))}\]
            \[\boxed{\displaystyle \frac{d}{dx}f(x) = 3\cos(x) + 2\sin(x)}\]
            \setcounter{enumi}{2}
      \item Differentiate.
            \[y = x^2 + \cot(x)\]
            \[{\displaystyle \frac{d}{dx}y = \frac{d}{dx}(x^2 + \cot(x))}\]
            \[{\displaystyle \frac{d}{dx}y = 2x - \csc^2(x)}\]
            \setcounter{enumi}{12}
            \newpage
      \item Differentiate.
            \[f(\theta) = \frac{\sin(\theta)}{1 + \cos(\theta)}\]
            \[\frac{d}{d\theta}f(\theta) = \frac{d}{d\theta}(\frac{\sin(\theta)}{1 + \cos(\theta)})\]
            \[\frac{d}{d\theta}f(\theta) = \frac{\frac{d}{d\theta}\sin(\theta)(1+\cos(\theta)) - (\sin(\theta))\frac{d}{d\theta}(1+cos(\theta))}{(1 + \cos(\theta))^2}\]
            \[\frac{d}{d\theta}f(\theta) = \frac{\cos(\theta)(1+\cos(\theta)) + \sin(\theta)\sin(\theta)}{(1 + \cos(\theta))^2}\]
            \[\frac{d}{d\theta}f(\theta) = \frac{\cos(\theta)+\cos^2(\theta) + \sin^2(\theta)}{(1 + \cos(\theta))^2}\]
            \[\frac{d}{d\theta}f(\theta) = \frac{1+\cos(\theta)}{(1 + \cos(\theta))^2}\]
            \[\boxed{\frac{d}{d\theta}f(\theta) = \frac{1}{1 + \cos(\theta)}}\]
            \setcounter{enumi}{18}
      \item Differentiate.
            \[y = \frac{t\sin(t)}{1 + t}\]
            \[\frac{d}{dt}y = \frac{d}{dt}(\frac{t\sin(t)}{1 + t})\]
            \[\frac{d}{dt}y = \frac{\frac{d}{dt}(t\sin(t))(1+t) - (t\sin(t))\frac{d}{dt}(1+t)}{(1 + t)^2}\]

\end{enumerate}

\end{document}